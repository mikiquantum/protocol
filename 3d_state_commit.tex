% describe how a document root hash is calculated
%---------------------------------------------------
%----- State Commitment
%---------------------------------------------------
\subsection{State commitment}

\subsubsection{Anchors}
An anchor is the root hash of a document $R_{\texttt{doc-root}}$ stored in an Ethereum smart contract called \textit{AnchorRepository}. 
\\\\
Every new document version $d'$ must commit its  document root hash $R_{\texttt{doc-root}}$ to the registry. A new document $d'$ is only valid and accepted if the $R_{\texttt{doc-root}}$ exists as an anchor in the registry.\\
%\textcolor{gray}{TODO: introduce blockheight}
\begin{eqnarray}
A &=& (A_{\texttt{anchor-id}}, A_{\texttt{anchor-root}}, A_{\texttt{anchor-blockheight}})\\
A_{\texttt{anchor-id}} & = & d_{\texttt{current}} \\
A_{\texttt{anchor-root}} & = & R_{\texttt{doc-root}} \\
A_{\texttt{anchor-blockheight}} & = & \texttt{block.number}
\end{eqnarray}
\\
The anchor id of a new document $d_{[i]{\texttt{current}}}$ is defined by the previous document $d_{[i-1]\texttt{next-img}}$. The same image is stored in the current 
document in $d_{[i]\texttt{current-img}}$. The hash of the $d_{[i]\texttt{current-img}}$ defines the current document identifier and the anchor id.
\begin{eqnarray}
d_{[i]\texttt{current}}& = & \mathtt{blake2b}(d_{[i-1]\texttt{next-img}})\\
A_{\texttt{anchor-id}}& = & d_{[i]\texttt{current}}
\end{eqnarray}
The \texttt{block.number} contains the Ethereum block number the transaction got mined and acts as a timestamp for the anchor.
\subsubsection{Verifying a document}
Any user can verify the correctness of the document by comparing the root hash to the committed hash on chain. We formally define a method called $\mathsf{getAnchor}$ which returns an anchor from the contract.
\begin{eqnarray}
A_{\texttt{anchor-current}} & \xleftarrow[]{\text{ETH}}& \mathsf{getAnchor}(d_{[i]{\texttt{current}}}) \\
A_{\texttt{anchor-next}} & \xleftarrow[]{\text{ETH}} & \mathsf{getAnchor}(d_{[i]{\texttt{next}}}) \\
\end{eqnarray}

\subsubsection{Latest Version}
Because in each document (state) update the next anchor id is already defined, anyone can verify that they have the latest version by checking the anchor registry. A document $d_{[i]}$ is the latest version if the following condition is true:
\begin{equation}
A_{\mathtt{anchor-current}} = R_{\mathtt{doc-root}} \quad \wedge \quad A_{\mathtt{anchor-next}} = 0
\end{equation}
If an anchor for $A_{\mathtt{anchor-next}}$ exists, the current document $d_{[i]}$ is not the latest version. It is not possible to overwrite an existing anchor id or invalid it. For committing an anchor the $d_{[i]\texttt{current-img}}$ is needed therefore only document collaborators could propose a next version.

\subsubsection{Pre-Commit}
A document is committed only after reaching consensus with the other collaborators on the document. Consensus is reached by getting a cryptographic signature from other parties by sending them the signing root. To deny the counter-party the free option of publishing its own state commitment upon receiving a request for signature, the node can first publish a pre-commit.
A pre-commit locks a commit to a specific document state for a given anchor id for a predetermined number of blocks. Only the \texttt{msg.sender} of the pre-commit is allowed to commit a corresponding anchor before the pre-commit has expired.

\begin{equation}\\
    \mathsf{preCommit}(d_{[i]\texttt{current}},R_{\texttt{signing}}) \xrightarrow[]{\text{ETH}} \mathsf{AnchorRepository}
    %TODO: leftover? M_{\texttt{proof}}(R_{\texttt{doc-root}})
\end{equation}\\
The pre-commit stores the provided signing root $R_{\texttt{signing}}$ on the smart contract. Only a $R_{\texttt{doc-root}}$ in the commit which includes the same $R_{\texttt{signing}}$ is accepted as a valid commit. 
It is not possible to provide a new preCommit $R_{\texttt{signing}}$ within the next  blocks for the given anchorId.\\
In the event that the user does not provide a commit within the lock time, the commit will be unlocked, and the \texttt{preCommit} method will be open to everyone again.

\subsubsection{Commit}
The anchor registry's \texttt{commit} method does not have any restrictions, except for the pre-commit flow as described above. If no valid pre-commit exists, anyone can commit an anchor to the registry. A commit of a new anchor is described by the following function. 
\begin{equation}\\
    \mathsf{commit}(d_{[i]\texttt{current-img}},R_{\texttt{doc-root}},P_{\texttt{sign-proof}}) \xrightarrow[]{\text{ETH}} \mathsf{AnchorRepository}
    %M_{\texttt{proof}}(R_{\texttt{doc-root}})
\end{equation}\\
\textbf{Next Pre-Image }\\
The commit method hashes $d_{[i−1]\texttt{next-img}}$ inside the smart contract to get the $A_{\texttt{anchor-id}}$. The $d_{[i−1]\texttt{next-img}}$ of the previous version is needed to commit the next version. It is not possible to use $d_{[i−1]\texttt{next}}$ (equal to $d_{[i]\texttt{current}}$) directly. In some cases, the field  $d_{\texttt{next}}$ is revealed to the outside world to prove the latest version of a document. (See \ref{sec:nft}). The anchor contract challenges the submitter to reveal the pre-image to avoid an attacker committing an malicious anchor for $d_{\texttt{next}}$. The field $d_{\texttt{next-img}}$ is never revealed to the outside world. \\
\\
\textbf{Signing Root Proof }\\
In case a pre-commit exists for the \texttt{anchorId}, calculated by hashing the given pre-image, the \texttt{msg.sender} must be the same as in the pre-commit. In the pre-commit, the signing-root $R_{\texttt{signing}}$ has been stored. A merkle proof verifies if the $R_{\texttt{signing}}$ is part of the $R_{\texttt{doc-root}}$ tree. $P_{\texttt{sign-proof}}$ is an array of $\mathbb{B}_{32}$ arrays which contains the needed sibling hashes for the merkle proof. 


%TODO
%\begin{eqnarray}
%\mathsf{commit}(d_{[i-1]\texttt{next-img}},R_{\texttt{doc-root}}) \\
%\mathsf{getAnchor}(d_{[i]{\texttt{current}}}) \to A_{\texttt{anchor}} \\
%\mathsf{getAnchor}(d_{[i]{\texttt{next}}}) \to A_{\texttt{anchor-next}} \\
%\sigma_{t+1} = T(\sigma_{t},d_{[i]},A_{\texttt{anchor}},A_{\texttt{anchor-next}})
%\end{eqnarray}



%\subsubsection{Commit old version}
%A node can commit to a state by publishing the root hash of the document on a contract. The document has a 32 byte randomly generated identifier, $I$ under which the commitment \math{c} is published. 

%TODO (Manuel): C used for something other than the C above (CoreDoc Fields)


%\begin{equation}
%c = C(I, RH(D)
%\end{equation}

%The state commitment function is a transaction on the Ethereum \texttt{AnchorRepository} contract\footnote{https://github.com/centrifuge/centrifuge-ethereum-contracts/blob/develop/contracts/AnchorRepository.sol}. It has a list of previous commitments and allows addition but not modification of commitments. We call these commitments \textit{anchors}.

 
