% TODO (Manuel): copy byte sequence definition from ethereum yellow paper
% TODO (Manuel): Define cryptographic hash function
% TODO (Lucas): we should add a section that introduces documents etc.
\section{Document and state commitments}\label{sec:doc_state_comm}
The Centrifuge protocol does not have the concept of a global state for documents. Instead, the state is only shared across collaborators of a single \textit{documents} which is the key component in the protocol.
A document has a predefined schema to represent a specific document type, like a purchase order or an invoice. A Centrifuge node can create a document and share it with others. The transport of a document happens privately over secure channels in the P2P network. Every document collaborator keeps a local copy of a document in their storage. An update of a document can be triggered by multiple collaborators and is not restricted to the initial document creator. Whenever a change is made, a
calculated merkle root of a document is committed on chain. A new version is only accepted by others if the document root hash exists in the \textit{AnchorRepository} and the set of mandatory document fields in the new version satisfy protocol-specific requirements.  A formal definition of the individual document components should expound the requirements of a valid new document state. 
\subsection{Conventions}
We use the following conventions in the document. The lowercase letter $d$ indicates a document in Centrifuge. Together with a subscript lowercase letter a field in the document or in a set is described. For example $d_{\texttt{id}}$ refers to  the \textsc{ID} field in a document $d$.  Lowercase letters with square brackets indicate an index. For example $d_{[2]}$ refers to the third document in a row. Arrows are used to describe communication with Ethereum and other participants in the P2P network. The arrow $\xrightarrow[]{\text{ETH}}$ is used to refer to Ethereum and $\xrightarrow[]{\text{P2P}}$ for the P2P network communication. For example $a \xrightarrow[]{\text{P2P}} b$, the right arrow describes a message send from a to b over the P2P network. A left arrow would indicate a P2P request or a call on an Ethereum smart contract (For example:  $a \ethleftarrow b$).
\subsection{Document}
%---------------------------------------------------
%----- Fields
%---------------------------------------------------
\subsubsection{Fields}
%TODO we need to define that the schema version is defined in the coredoc
The smallest unit of a document is called \textit{field}.  
A field $F$ is a 3-tuple  with a field name defined as $F_\texttt{name}$, a field value defined as $F_\texttt{value}$ and a salt defined as $F_\texttt{salt}$
\begin{eqnarray}
F = (F_\texttt{name},F_\texttt{value},F_\texttt{salt}) \\
\begin{array}[t]{lclclc}
F_{\texttt{name}} \in \mathbb{B} \wedge F_{\texttt{value}} \in \mathbb{B} \wedge  F_{\texttt{salt}} \in \mathbb{B}_{32} \\
\end{array}
\end{eqnarray}
Elements of $\mathbb{B}$ are an arbitrary-length byte array. Elements of $\mathbb{B}_{32}$ are a 32 byte array. 

\subsubsection{Document}
A document $d$  can be formally expressed as the union of the schema field set $S$ with a set of core fields called core document denoted as $C$.


%TODO: add reference to centrifuge-protobufs

\begin{equation}
d = S \cup  C
\end{equation}
For referring to all schema fields $S$ of document or to the core document fields $C$ we introduce the following notation
\begin{eqnarray}
d_S = S \\
d_C = C
\end{eqnarray}

\subsubsection{Schema Fields}\label{sec:doc_schema_fields}
A document $d$ contains a set of fields $S$ described by a document schema. The document schema defines the type of a document. The document schema for an invoice would define the required fields for an invoice.
The set schema fields $S$ can be defined as
\begin{equation}
S = (F_{[0]},...,F_{[n]})\\
\end{equation}
In the set of specific fields $S$ for a document $d$, every field has a unique field name $F_{[i]\texttt{name}}$ to identify a member of the set.
\begin{equation}
F_{[i]} \in  S \Rightarrow F_{[i]\texttt{name}} \neq  F_{[j]\texttt{name}} \, ; \forall F_{[j]}  \in S, i \neq j 
\end{equation}
All participants in the protocol agree in advance on the schema of documents the would like to exchange. In addition to the predefined schema fields a schema could contain a set of customized attributes fields for required additional information.

%\begin{equation}
%\{x \in \{0..n\}| \nexist y \in \{0..n\}\x: F[x]_i=F[y]_i \land x & \neg & y \}
%\end{equation}

\subsubsection{Core Document Fields\label{Core Document}}\label{sec:doc_core_doc_fields}
The core document $C$ defines the required fields for every document $d$.
The core document contains signatures, hashes, and other relevant information that allows users to exchange any document data.
 
We introduce the fields in $C$ step by step in the following section and define the required constraints of the core document fields. Variables denoted with $R_{\texttt{x}}$ are derived from other fields but are themselves not persisted on the document. $R_{\texttt{x}}$ variables are needed for signing and for anchoring documents in the smart contract. A document field which is a core document field is denoted with $d_{\texttt{x}}$. 
The core document\footnote{\url{https://github.com/centrifuge/centrifuge-protobufs/blob/master/coredocument/coredocument.proto}} includes other fields not mentioned in this section which are introduced in the following sections step by step.
\begin{description} 
\item{\textbf{Identifier}} a $\mathbb{B}_{32}$ value defines the unique identifier of a document, formally $d_{\texttt{id}}$ 
\item{\textbf{Current Version}} a $\mathbb{B}_{32}$ value defines a unique id for the current version of a document, formally as $d_{\texttt{current}}$
\item{\textbf{Current Version Pre Image}} is a $\mathbb{B}_{32}$ value needed to define the current version of a document,  formally as $d_{\texttt{current-img}}$
\item{\textbf{Previous Version}} a $\mathbb{B}_{32}$ value needed to refer to the previous version of a document, formally as $d_{\texttt{prev}}$
\item{\textbf{Next Version Pre Image}} is a randomly chosen $\mathbb{B}_{32}$ value needed to define the next version of a document. This should be kept private to collaborators who can update the state. Formally as $d_{\texttt{next-img}}$
\item{\textbf{Next Version}} a $\mathbb{B}_{32}$ value defines a unique identifier for the next version of a document, formally $d_{\texttt{next}}$
\item{\textbf{Document Root}} a $\mathbb{B}_{32}$ value defines a root hash for document, formally $R_{\texttt{doc-root}}$
\item{\textbf{Signing Root}} a $\mathbb{B}_{32}$ value defines the root hash of the Merkle tree containing all fields except for the signatures, formally $R_{\texttt{signing}}$
\item{\textbf{Signatures}} a set of signatures of the signing root denoted as $d_{\texttt{signatures}}$
\item{\textbf{Previous Root}} a $\mathbb{B}_{32}$ value defines the document hash of the previous version of the document. The previous root denoted as $d_{\texttt{prev-root}}$
\item{\textbf{Nonce}} is a $\mathbb{B}_{32}$ value needed to indicate the version history of a document,  formally as $d_{\texttt{nonce}}$
\end{description}

